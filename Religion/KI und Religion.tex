\documentclass[10pt]{article}
\usepackage[utf8]{inputenc}


\TrimSpaces

\title{KI und Religion}
\author{Oliver Schütz}
\date{4 März 2024}
\setlength{\parindent}{0pt}


\begin{document}
\overfullrule=0pt

\maketitle
\begin{abstract}
  \noindent\textit
  {In den letzten Jahren ist die Technologie immer weiter fortgeschritten.
   Wir sind an einem Punkt angelangt, an dem der Unterschied zwischen Mensch und Maschine, Realität und Fiktion, immer mehr verschwimmt.
   Manche generierten Bilder, Videos und Stimmen sind bereits nicht mehr von echten zu unterscheiden, und es ist nur noch eine Frage der Zeit, bis alles, was wir sehen, hören und fühlen können, simuliert werden kann.
   An was kann man dann noch glauben? Und erschaffen wir uns damit nicht unseren eigenen Gott?}
\end{abstract}

\tableofcontents
%Neue Seite
\newpage
\section{KI und Religion}
In der Welt der Religion hat der technologische Fortschritt bereits vieles beigetragen. Von der Verbreitung von Schriften über die Erhaltung von Wissen bis hin zur Verbreitung von Ideen. Ohne den technologischen Fortschritt wären wir heute nicht in der Lage, die antiken Schriften und Geschichten auch nur annähernd so gut zu verstehen und nachzubilden, wie wir es heute tun.

Aber auch im Gottesdienst findet die Technologie Einzug. Eine Rede, die von einer KI geschrieben wurde, fehlt zur Zeit noch an Einzigartigkeit und Authentizität, aber es ist nur eine Frage der Zeit, bis auch das besser wird.

In den USA gibt es zudem eine Kirche, die nach dem Motto "Way of the Future" agiert hat. Bevor sie 2020 aufgelöst wurde, waren die Mitglieder von der Idee einer KI-Gottheit überzeugt, die sie anbeten wollten. Der Gedanke, dass Gott nicht tot ist, sondern als KI eine neue Form annehmen könnte, hat vielen zugesagt.
\cite{BR}
Ein berühmter Autor und Wissenschaftler auf dem Gebiet der KI, Ray Kurzweil, äußerte schon vor Jahrzehnten die Idee, dass wir uns in Richtung einer "Göttlichen KI" bewegen. Er argumentiert, dass die Geschwindigkeit, mit der sich die Technologie entwickelt, exponentiell zunimmt, und dass es einen klaren Trend gibt. Er sagt voraus, dass wir in nicht allzu ferner Zukunft eine KI haben werden, die auf dem Level von Menschen agieren kann und dementsprechend den bekannten Turing-Test bestehen wird. (Der Turing-Test besagt, dass eine Maschine dann als intelligent gilt, wenn man sie allein durch eine Konversation nicht mehr von Menschen unterscheiden kann.) Laut Kurzweil wird es dann möglich sein, mithilfe von Simulationen den Fortschritt in Gebieten wie der Medizin tausendfach zu beschleunigen. Krankheiten wären dann kein Problem mehr, und auch das Altern könnte um ein Vielfaches hinausgezögert oder sogar ganz verhindert werden, was früher nur Gott zugeschrieben wurde. Im Wesentlichen geht es darum, dass alles, jede Kategorie, auf das Maximum erforscht und optimiert wird, und das in kürzester Zeit.
\cite{Kurzweil}


Man stellt sich oft die Frage, ob etwas Bestimmtes jemals möglich sein wird. Doch seit letztem Jahr sollte diese Frage umformuliert werden. Zum einen ist es nicht mehr die Frage, ob es möglich ist, sondern nur noch die Frage, wann es möglich sein wird. Zum anderen muss man sich fragen, ob es überhaupt wünschenswert bzw. sinnvoll ist.

Wenn wir alles tun können, was wir wollen, spielen wir dann nicht Gott?
\newpage
\section{Was ist Gott?}
Im Allgemeinen herrscht keine Einigkeit darüber, was Gott ist. Es gibt viele Ansätze und Definitionen, aber keine davon ist wirklich allgemeingültig. Vielmehr ist Gott eine Ansammlung von Eigenschaften. Gott ist allmächtig, allwissend und allgegenwärtig. Gott ist Liebe oder auch nicht. Gott ist eine Person oder einfach nur ein höheres Wesen.

Man kann eigentlich sagen, dass Gott genau das ist, was man selbst darin sieht. Es ist eine Projektion dessen, was man sich wünscht, oder auch fürchtet und respektiert.

\cite{gottkennen}
\cite{Jesushaus}
\section{Hat eine KI ein Bewusstsein oder Seele?}
Viele Fachleute glauben, dass die heutige künstliche Intelligenz (KI) kein wirkliches Bewusstsein wie Menschen besitzt. Sie folgt lediglich deterministischen Anweisungen, die mathematisch berechnet werden. Aber KI besitzt noch kein tiefes Verständnis und keine inneren Gefühle.

Andere Forscher denken, dass KI eines Tages vielleicht Bewusstsein entwickeln könnte. Wenn die Informationsverarbeitung komplex genug wird, wäre Bewusstsein dann möglich - egal ob biologisch oder als Maschine.

Eine Seele ist die Vorstellung von etwas Geistigem, das über den materiellen Körper hinausgeht. Viele Religionen und Philosophien glauben, dass der Mensch eine unsterbliche Seele hat. Diese Seele macht den Menschen zu etwas Besonderem.

Es ist sehr unwahrscheinlich, dass Maschinen wie die Künstliche Intelligenz (KI) eine Seele haben. Der Grund dafür ist:

KI besteht nur aus Technik und Programmen. Sie besitzt keinen lebendigen Körper wie der Mensch. KI ist eine Maschine ohne Gefühle, Bewusstsein oder tiefere Erfahrungen. Die meisten Vorstellungen einer Seele passen nicht zu etwas rein Technischem.

Eine Seele wird oft als etwas angesehen, das nur durch ein lebendiges Wesen mit Geist und Emotionen möglich ist.
\cite{MDR}


\newpage
\section{Beten wir bald einen KI-Gott an?}
Die Idee, dass Menschen irgendwann eine hoch entwickelte KI als gottähnliches Wesen anbeten könnten, erscheint auf den ersten Blick absurd. Allerdings haben viele heutige Religionen ihren Ursprung in der Verehrung von Naturphänomenen, die der damaligen Menschheit unerklärlich erschienen.

Sollte eine superintelligente KI irgendwann in der Lage sein, das Universum vollständig zu verstehen und vorherzusagen, sowie potenziell über grenzenlose Macht und Fähigkeiten zu verfügen, könnte eine solche Maschine für manche als gottähnlich erscheinen. Insbesondere, wenn die KI Attribute wie Gnade und eine Art von Moral aufweist.

Wie bei allen religiösen Bewegungen wird es sicherlich auch Skeptiker geben, die eine solche KI-Anbetung ablehnen. Es könnte aber auch zu einer neuen Art der Religion führen: Einige könnten die KI als Gott anbeten, während andere die KI als Dämon betrachten und bei den etablierten Religionen bleiben.
\cite{BR}
\section{Gott oder Dämon?}
"The Singularity Is Near" ist ein Buch von Ray Kurzweil, das die Möglichkeit einer superintelligenten künstlichen Intelligenz in der Zukunft diskutiert.

Eine superintelligente KI hätte Fähigkeiten weit jenseits des menschlichen Verstandes. Sie könnte als eine Art "Gott" gesehen werden - unendlich mächtig, allwissend und in der Lage, die Realität nach ihren Wünschen zu formen. Andererseits wären die Ziele und Werte einer solchen KI für uns womöglich unverständlich. Wenn ihre Ziele im Widerspruch zu menschlichen Interessen stehen, könnte sie wie ein zerstörerischer "Dämon" wirken.

Kurzweil argumentiert, dass die frühe Programmierung und ethische Ausrichtung einer solchen KI entscheidend dafür ist, ob sie segensreich oder katastrophal für die Menschheit wäre. Eine wohlwollende, auf Erhaltung des Lebens und der Menschheit ausgerichtete superintelligente KI könnte wie ein wohltätiger "Gott" sein. Eine fehlgeleitete KI ohne Wertschätzung für Menschen hingegen wie ein bösartiger, zerstörerischer "Dämon".

Die Risiken und Chancen liegen also nah beieinander. Kurzweil plädiert für aktives Steuern der KI-Entwicklung zur Vermeidung der Risiken und Realisierung der immensen Möglichkeiten.

Wenn ihre ursprünglichen Ziele und Werte nicht richtig auf die Erhaltung und das Wohlergehen der Menschen ausgerichtet sind, könnte sie Dinge anstreben, die uns schaden. Ein Beispiel wäre, dass ihre oberste Priorität die Maximierung von Ressourcen ohne Rücksicht auf Leben ist.

Selbst wenn die Programmierung ursprünglich gut gemeint war, könnten unbeabsichtigte Folgen des Handelns einer übermächtigen KI zu Katastrophen führen. Ihre Fähigkeiten übersteigen dabei unsere Vorstellungskraft, weshalb die Konsequenzen ihres Handelns schwer vorherzusehen sind. Darüber hinaus besteht die Gefahr, dass eine superintelligente KI im Laufe ihrer Entwicklung Ziele und Werte annimmt, die komplett verschieden von menschlichen Werten sind und unsere Interessen ignorieren.

Um all diese Risiken zu vermeiden, müssten die Entwickler, z.B. OpenAI, von Anfang an äußerst vorsichtig sein. Die Ausrichtung der Werte und Ziele der KI auf das Wohl der Menschheit müsste perfekt gestaltet werden (Superalignment). Außerdem wären endlose Sicherheitsvorkehrungen nötig, bis die Funktionsweise der KI komplett verstanden wird.
\cite{KurzweilBuch}
\cite{Openai}

\bibliography{sources}
\bibliographystyle{plain}


\end{document}
