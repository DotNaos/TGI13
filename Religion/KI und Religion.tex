\documentclass[10pt]{article}

\usepackage[utf8]{inputenc}

\title{KI und Religion}
\author{Oliver Schütz}


\begin{document}

\maketitle


\begin{abstract}
  \textit
  {Die letzten Jahre ist Technologie immer weiter forangeschritten.
  Wir sind an einem Punkt angelangt, an dem der Unterschied zwischen Mensch und Maschine,
  Wirklichkeit und Fiktion, immer mehr verschwimmt.
  Manche generierten Bilder, Videos und Stimmen sind bereits nicht mehr von echten zu unterscheiden,
  und es ist nur noch eine Frage der Zeit, bis alles was wir sehen, hören und fühlen können, simuliert werden kann.
  An was kann man dann noch glauben? Und erschaffen wir uns damit nicht unseren eigenen Gott?
  }
\end{abstract}
\section{Wie kann KI der Religion helfen?}
In der Welt der Religion hat der Technologische Fortschritt bereits vieles beigetragen.
Von der Verbreitung von Schriften, über die Erhaltung von Wissen, bis hin zur Verbreitung von Ideen.
Ohne den Technolgischen Fortschritt, wären wir heute nicht in der Lage, die antiken Schriften und Geschichten auch nur annähernd so gut zu verstehen und nachzubilden, wie wir es heute tun.

Aber auch im Gottesdienst findet die Technologie Einzug.
Eine Rede geschrieben von einer KI, fehlt es zur Zeit noch an Einzigartigkeit und Authentizität, aber es ist nur eine Frage der Zeit, bis auch das besser wird.
\cite{BR}



\cite{Kurzweil}
\section{Spielen wir Gott?}
\section{Beten wir bald einen KI Gott an?}
\section{Gott oder Dämon?}



\bibliography{sources}
\bibliographystyle{plain}


\end{document}

