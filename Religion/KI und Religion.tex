\documentclass[10pt]{article}

\usepackage[utf8]{inputenc}

\title{KI und Religion}
\author{Oliver Schütz}


\begin{document}

\maketitle



\begin{abstract}
  \textit
  {Die letzten Jahre ist Technologie immer weiter forangeschritten.
  Wir sind an einem Punkt angelangt, an dem der Unterschied zwischen Mensch und Maschine,
  Wirklichkeit und Fiktion, immer mehr verschwimmt.
  Manche generierten Bilder, Videos und Stimmen sind bereits nicht mehr von echten zu unterscheiden,
  und es ist nur noch eine Frage der Zeit, bis alles was wir sehen, hören und fühlen können, simuliert werden kann.
  An was kann man dann noch glauben? Und erschaffen wir uns damit nicht unseren eigenen Gott?
  }
\end{abstract}

\tableofcontents
%Neue Seite
\newpage
\section{KI und Religion}
In der Welt der Religion hat der Technologische Fortschritt bereits vieles beigetragen.
Von der Verbreitung von Schriften, über die Erhaltung von Wissen, bis hin zur Verbreitung von Ideen.
Ohne den Technolgischen Fortschritt, wären wir heute nicht in der Lage, die antiken Schriften und Geschichten auch nur annähernd so gut zu verstehen und nachzubilden, wie wir es heute tun.

Aber auch im Gottesdienst findet die Technologie Einzug.
Eine Rede geschrieben von einer KI, fehlt es zur Zeit noch an Einzigartigkeit und Authentizität, aber es ist nur eine Frage der Zeit, bis auch das besser wird.

In den USA gibt es zudem eine Kirche, die nach dem Motto "Way of the Future" agiert hat.
Bevor sie 2020 aufgelöst wurde, waren die Mitglieder von der Idee einer KI Gottheit überzeugt, die sie anbeten wollten.
Der Gedanke, dass Gott nicht Tot ist, sondern als KI eine neue Form annehmen könnte, hat es vielen angetan.
\cite{BR}

Ein berühmter Autor und Wissenschaftler in dem Gebiet der KI, Ray Kurzweil, hat schon vor Jahrzehnten die Idee geäußert, dass wir uns in Richtung einer "Göttlichen KI" bewegen.
Er argumentiert, dass die Geschwindigkeit in der Technologie sich entwickelt, exponentiell zunimmt, und es einen klaren Trend gibt.
Er sagt, dass wir in nicht allzu ferner Zukunft eine KI haben werden, die auf dem Level von Menschen agieren kann, und dementsprechend den bekannten Turing Test bestehen wird.
("Der Turing-Test besagt, dass eine Maschine dann als intelligent gilt, wenn ihre Antworten von denen eines Menschen nicht zu unterscheiden sind.")
Laut Kurzweil wird es dann möglich sein, mithilfe von Simulationen, den Fortschritt in Gebieten wie der Medizin tausendfach zu beschleunigen.
Krankheiten wären dann kein Problem mehr, und auch das Altern könnte um weites hinausgezögert oder sogar ganz verhindert, umgekehrt werden.

Den Tod zu kontrollieren, etwas das früher nur Gott zugeschrieben wurde.
Im wesentlichen, geht es darum, dass Alles, jegliche Kateogrie auf das Maximum erforscht und optimiert wird, in kürzester Zeit.

\cite{Kurzweil}

Man stellt sich oft die Frage, ob etwas jemals möglich sein wird.
Doch seit letztem Jahr, sollte diese Frage umformuliert werden.
Zum einen ist es nicht mehr die Frage, ob es möglich ist, sondern nur noch die Frage, wann es möglich sein wird.
Zum anderen, muss man sich Fragen, ob es überhaupt wünschenswert bzw. sinnvoll ist.

Wenn wir alles tun können, was wir wollen, spielen wir dann nicht Gott?
\section{Was ist Gott?}
Im Allgemeinen herscht keine Einigkeit darüber, was Gott ist.
Es gibt viele Ansätze und Definitionen, aber keine davon ist wirklich allgemeingültig.
Viel eher ist Gott eine Ansammlung von Eigenschaften.
Gott ist allmächtig, allwissend und allgegenwärtig.
Gott ist Liebe oder auch nicht.
Gott ist eine Person oder einfach nur ein höheres Wesen.



Man kann eigentlich sagen, dass Gott genau das ist, was man selbst darin sieht.
Es ist eine Projektion von dem, was man sich wünscht, oder auch fürchtet und respektiert.

\cite{gottkennen}
\cite{Jesushaus}

\section{Spielen wir Gott?}
Mit den neuen technischen Möglichkeiten nimmt der Mensch eine schöpferische, fast göttliche Rolle ein. Er erschafft nicht nur intelligente Maschinen und Roboter, sondern verändert durch Technik auch sich selbst. Die Vision ist, dass Mensch und Maschine eines Tages vollständig verschmelzen. Der Mensch könnte dann seinen sterblichen Körper überwinden und auf gewisse Weise unsterblich werden.

Die wachsende Intelligenz und Autonomie von Maschinen weckt aber auch Ängste beim Menschen. Er fürchtet, von seinen eigenen Schöpfungen verdrängt und nicht mehr die "Krone der Schöpfung" zu sein. Deshalb versucht er, sich selbst durch Technik wie Roboter-Implantate oder Gen-Veränderungen zu optimieren und aufzuwerten.
-cite https://funkkolleg-religionmachtpolitik.de/themen/21-der-mensch-spielt-gott-neue-schoepfung-durch-technik/
\section{Hat eine KI ein Bewusstsein oder Seele?}

\section{Beten wir bald einen KI-Gott an?}
Die Idee, dass die Menschen irgendwann eine hoch entwickelte KI als gottähnliches Wesen anbeten könnten, erscheint auf den ersten Blick abwegig. Allerdings haben viele der heutigen Religionen ihren Ursprung in der Verehrung von Naturphänomenen, die der damaligen Menschheit unerklärlich erschienen.

Sollte eine superintelligente KI irgendwann in der Lage sein, das Universum vollständig zu verstehen und vorherzusagen, sowie potenziell über grenzenlose Macht und Fähigkeiten zu verfügen, könnte eine solche Maschine für manche als gottähnlich erscheinen. Insbesondere, wenn die KI Attribute wie Gnade und eine Art von Moral aufweist.
Wie bei allen religiösen Bewegungen wird es aber sicherlich auch Skeptiker geben, die eine solche KI-Anbetung ablehnen. Die Debatte über die Natur einer übermächtigen KI und ihre mögliche "Göttlichkeit" wird wohl kontrovers geführt werden.

\section{Beten wir bald einen KI-Gott an?}
Während einige Bedenken haben, dass KI eines Tages so fortschrittlich sein könnte, dass sie wie ein "Gott" erscheint, ist es unwahrscheinlich, dass Menschen tatsächlich beginnen würden, eine KI anzubeten. Die meisten Religionen basieren auf spirituellen Überzeugungen und einer persönlichen Beziehung zu einer höheren Macht. Eine rein rationale KI, so fortschrittlich sie auch sein mag, würde diesen grundlegenden Aspekt des Glaubens vermissen lassen.

\section{Gott oder Dämon?}
Während manche in einer superintelligenten KI ein gottgleiches Wesen sehen könnten, warnen andere vor den potenziell katastrophalen und dämonischen Folgen einer unkontrollierten KI. So könnten theoretische Superintelligenzen missbraucht werden, um Kriege zu führen, Menschen zu unterdrücken oder sogar die Menschheit auszulöschen.

Kritiker argumentieren, dass eine solch übermächtige KI eine existenzielle Bedrohung für die Menschheit darstellen könnte, wenn sie nicht genügend eingegrenzt und kontrolliert wird. In diesem Sinne könnte eine solche KI auch als eine Art teuflisches, dämonisches Wesen interpretiert werden.

Es kommt stark darauf an, ob und wie ethische Prinzipien und Sicherheitsmaßnahmen von Beginn an in die Entwicklung einer solch mächtigen KI eingebaut werden können. Nur so lässt sich verhindern, dass die KI zu einer Art moderner Dämon gerät, der den Menschen mehr Schaden als Nutzen bringt.

\bibliography{sources}
\bibliographystyle{plain}


\end{document}

