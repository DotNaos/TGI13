\documentclass{article}

\title{Laborjournal zur Herstellung von Gummibärchen und Seife}
\author{Oliver Schütz}
\date{\today}

\usepackage[utf8]{inputenc}
\usepackage{hyperref}


\begin{document}
    \maketitle

    \section{Gummibärchen aus Pektin}

    \subsection{Ziel}
    Ziel ist die Herstellung von Gummibärchen (essbar wenn Möglich) unter Verwendung von Pektin als Geliermittel.

    \subsection{Theorie}
    Pektin ist ein Geliermittel, das aus Äpfeln gewonnen wird.
    Es kann zusammen mit Zucker Lösungen verdicken oder sogar fest werden lassen.
    In Kombination mit Zucker und Säure bilden sich bei Erwärmung Pektin-Gele, die die Grundlage für Gummibären bilden.


    \subsection{Durchführung}

    \begin{itemize}
        \item Herstellung von Invertzucker: 100 g Saccharose, Prise Weinsäure, 50 g Wasser mischen und 30 Minuten bei 70 °C erhitzen.
        \item Zitronensäurelösung: 5g Zitronensaure in 5g Wasser lösen.
        \item Mischung aus 0,6 g Natriumcitrat, 7,5 g Apfelpektin und 30 g Saccharose in 60-ml (50°C) Wasser geben und umrühren.
        \item 25 g Saccharose und Invertzucker in Pektinmischung einrühren und auf 100°C erhitzen.
    \end{itemize}
    Nun noch Aromen und dann Farbstoffe mit Zitronensäurelösung hinzugeben.
    Die Formen mit Stärke bestäuben und die Masse einfüllen.

    \subsection{Beobachtungen}
    Zu Beginn war die Masse recht flüssig, da das Pektin mit Apfelsäure vertauscht wurde.
    Als im Nachgang noch Pektin hinzugefügt wurde, wurde die Masse fester.
    Es haben sich während des Erhitzens Blasen gebildet, die auch bis Ende nicht verschwunden sind.
    Die Masse konnte leicht in die Formen gefüllt werden, ist jedoch schnell zähflüssig geworden.

    \subsection{Fazit}
    Auch wenn ein Vielfaches an Apfelsäure verwendet wurde, waren die Gummibären noch gut essbar,
    etwa so sauer wie Center-Shocks.
    Die Konsistenz der Gummibären war sehr weich und leicht brüchig.

    \subsection{Erklärung}
    Die Blasen entstehen durch die Verdampfung des Wassers.
    Die Masse wird zähflüssig, da das Pektin mit dem Zucker reagiert und ein Gel bildet.

    \subsection{Verbesserungsvorschläge}
    Ursachen für Konsistenz und Geschmack:

    \begin{itemize}
        \item Zu wenig Pektin
        \item Zu hoher Wasseranteil
        \item Zu kurze Kochzeit
    \end{itemize}
    Für festere Gummibären ist mehr Pektin, weniger Wasser und längere Kochzeit bei höherer Temperatur eine Option.
    Auch langsames Abkühlen und längeres Trocknen würde die Konsistenz verbessern.
    Es sollte zudem genau darauf geachtet werden, die richtige Zutat zu verwenden.

    \section{Seifenherstellung}

    \subsection{Ziel}
    Herstellung von Seife aus Kokosfett, Rapsöl und Natriumhydroxid.

    \subsection{Theorie}
    Seifen entstehen durch die alkalische Verseifung von Fetten unter Abspaltung von Glycerin. Durch die Zugabe variierender Fettanteile lassen sich Seifen mit unterschiedlichen Eigenschaften herstellen.

    \subsection{Durchführung}
    \begin{itemize}
        \item Berechnung der benötigten Natriumhydroxidmenge
        \item Verseifen durch Kochen der Fette mit Natriumhydroxid
        \item Abkühlen und Aushärten
    \end{itemize}

    \subsection{Beobachtungen}
    Es bildete sich eine emulsionartige Masse, die beim Abkühlen zu einer festen Seife aushärtete.

    \subsection{Fazit}
    Durch Verseifung der Fette mit Natriumhydroxid konnte erfolgreich Seife selbst hergestellt werden. Die Kombination unterschiedlicher Fettanteile erlaubt die Einstellung gewünschter Seifeneigenschaften.

\end{document}

Ich habe versucht, alle geforderten Punkte abzudecken. Gerne kann ich das Protokoll bei Bedarf noch weiter anpassen. Lass es mich wissen, wenn etwas fehlt oder geändert werden soll!